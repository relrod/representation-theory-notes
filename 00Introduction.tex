\section{Introduction}
\subsection{Project Goals and Outline}

This project, proposed by Dr. Thomas Madsen, will provide an opportunity for
exploring two vast realms of mathematics: representation theory and category
theory.

We will begin by reviewing the necessary foundations for an understanding of
the basic results of representation theory. This includes a review of linear
algebra (vector spaces, linear maps), finite group theory (groups,
homomorphisms), ring and field theory (the axioms which comprise such objects),
and other notational components which may be found within the topic space.

After the review, we will set definitions and begin to explore the basic
results and connections found within representation theory. We will take a more
\textit{synthetic} approach to the definitions and theories within, focusing on
axioms where we can, and building up structure as we go.

Once the basis of representation theory has been established, we will lay out
a similar foundation for category theory. We will explore the possibility of
categorical connections and generalizations of results commonly used within
representation theory.

Towards the end of the present document, we will list some potentially
interesting research topics.

We will primarily be using the Benjamin Steinberg book,
\textit{Representation Theory of Finite Groups} \cite{steinberg} for the
representation theory presentation; the Joseph Gallian book,
\textit{Contemporary Abstract Algebra} \cite{gallian} for the abstract algebra
presentation, and (insert book here) for the category theory presentation.

\subsection{Silly Rambling and Side Goals}

In addition to the above, some interesting research could be done in connecting
representation theory to constructive mathematics and type theory. There
seems to be some limited research done in constructive representation theory,
specifically within Homotopy Type Theory (HoTT), a very new branch of
mathematics combining homotopy theory, type theory, and constructive
foundations. Particularly, the nLab page on representation theory
\cite{nlabRT} provides connections, stating, "The fundamental concepts of
representation theory have a particular natural formulation in homotopy theory
and in fact in homotopy type theory, which also refines it from the study of
representations of groups to that of $\infty$-representations of
$\infty$-groups." The page also provides a table showing the correspondence to
the formulation of key concepts in representation theory and in homotopy type
theory.
