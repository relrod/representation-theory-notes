\section{A Review of Abstract and Linear Algebra}

\subsection{Our goal: What is a representation?}
By the end of this review, we want to have enough pieces in place to understand
the definition of a representation.

\begin{defn}
  Let $G$ be a group. A \textit{representation} (see 3.1.1 in \cite{steinberg})
  of $G$ is a homomorphism $\varphi : G \to GL(V)$ for some finite-dimensional
  vector space $V$.
\end{defn}

To understand this definition, we need to understand the following terms:
\textit{group, homomorphism,} (finite-dimensional) \textit{vector space}. We
also need to understand how $GL(V)$ is a group.

% TODO: How much should we assume the reader knows? For now...
We assume a basic familiarity with the definition of \textit{group} and
\textit{homomorphism}.

\subsection{Vector Spaces}

We begin with the definition of a vector space over a field.

\begin{defn}
  Let $F$ be a field. A \textit{vector space} over $F$, then, is a set $V$ with
  two operations:
  \begin{align*}
    &+ : V \times V \to V && \text{(vector addition)}\\
    &\cdot : F \times V \to V && \text{(scalar multiplication)}
  \end{align*}
  such that vector addition forms an abelian group and the following equalities
  hold for $a, b \in F$ and $\textbf{u, v} \in V$:
  \begin{align*}
    &a(\textbf{v}+\textbf{u}) = a\textbf{v} + a\textbf{u} &&
    \text{(distributivity of scalar multplication over vector addition)}\\
    &(a+b)\textbf{v} = a\textbf{v} + b\textbf{v} && \text{(distributivity of
      scalar multplication over field addition)}\\
    &a(b\textbf{v}) = (ab)\textbf{v} && \text{(compatibility of field
      multplication and scalar multplication\footnotemark)}\\
    &1\textbf{v} = \textbf{v} && \text{(identity of scalar multiplication, where
      $1 \in F$)}
  \end{align*}
  \footnotetext{This does not assert the associativity of either operation.}
\end{defn}

\begin{note}
  In some texts, the term ``module'' is used. A \textit{module} over a ring is a
  generalization of a vector space over a field. That is, it allows $F$ in our
  definition above to be any ring, not specifically a field.

  Thus, if $K$ is a field, the following terms are equivalent: $K$-module,
  $K$-vector space, module over $K$, vector space over $K$.
\end{note}
